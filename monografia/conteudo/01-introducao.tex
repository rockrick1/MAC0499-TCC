%!TeX root=../tese.tex
%("dica" para o editor de texto: este arquivo é parte de um documento maior)
% para saber mais: https://tex.stackexchange.com/q/78101/183146

%% ------------------------------------------------------------------------- %%
\chapter{Introdução}
\label{cap:introducao}
\section{Motivação}

A dificuldade em um jogo tende a ser um elemento extremamente visível e importante para o seu aproveitamento. Jogos mais desafiadores são conhecidos por serem mais frustrantes, e os menos desafiadores, mais relaxantes. Porém, esse pensamento pode levar à falsa ideia de impossibilitação do aproveitamento por parte da dificuldade elevada. No artigo de Jesper~\citet{FearOfFailing} é discutido o verdadeiro papel da dificuldade em jogos, de forma enxuta e didática. Quando apresentamos o conceito de vitória em um jogo, não faria sentido termos este sem o conceito de derrota. O jogador não teria a sensação de superação no momento da vitória se o conceito de derrota não estivesse constantemente presente na experiência. A derrota é introduzida ao jogo de forma a induzir os usuários a reajustarem sua perspectiva perante ao jogo, como visto no estudo de Juul. O jogador é então levado a sempre pensar em novas estratégias e, consequentemente, adicionando conteúdo ao jogo.

O nível de dificuldade de um jogo pode ser visto como o responsável pelo desenvolvimento nas habilidades do jogador em experimentos como o de~\citet{ExperimentalValidation}, porém, neste mesmo experimento, vemos ela também influenciando fortemente o aproveitamento geral do jogo, e podendo prejudicá-lo se não definida apropriadamente. Segundo~\citet{VideoGameBusiness}, jogos eletrônicos consolidam uma das formas de entretenimento mais consumidas atualmente, totalizando uma receita de 159.3 bilhões de dólares no mundo inteiro em 2020. Portanto é natural associarmos jogos a entretenimento e, consequentemente, imaginarmos que um dado jogo é divertido, porém adequar a dificuldade de um jogo à experiência desejada prova-se um trabalho difícil.

\section{Objetivos}

Considerando a tese apresentada, é necessário manter em mente que um jogador pode se sentir \textquotedbl{violado}\textquotedbl{} pelo jogo se este sofre mudanças drásticas em sua dificuldade durante ou entre as sessões jogadas~\citep{DynamicDiffAdjustment}. Com isso, foi almejado um sistema que possibilitasse o uso de ajustes dinâmicos de dificuldade em um jogo, ao mesmo tempo que se adapta ao nível da habilidade do jogador. Isso é, se a habilidade do jogador se enquadra na dificuldade atual do jogo, esta tende a aumentar, propondo um desafio adequado à habilidade do jogador. Em contrapartida, se o jogador é muito prejudicado pela dificuldade atual do jogo, esta tende a diminuir, com o fim de manter o jogador em um estado mais favorável à sua capacidade atual, porém também o forçando a adaptar suas estratégias e seu estilo de jogo e não o impedindo de \textquotedbl{avançar}\textquotedbl{} na dificuldade de jogo.

Para este fim, foi desenvolvido um protótipo de jogo no estilo \textit{Shoot 'em Up} (\ref{genero}) através da plataforma de desenvolvimento \textit{Godot Engine} (\ref{cap:plataforma de desenvolvimento}), junto a um sistema de balanceamento dinâmico de dificuldade adequado ao jogo.

%\textquotedbl{aspa}\textquotedbl{}