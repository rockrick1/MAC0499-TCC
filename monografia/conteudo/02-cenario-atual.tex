%!TeX root=../tese.tex
%("dica" para o editor de texto: este arquivo é parte de um documento maior)
% para saber mais: https://tex.stackexchange.com/q/78101/183146

%% ------------------------------------------------------------------------- %%
\chapter{Introdução}
\label{cap:introducao}
\section{Motivação}

A dificuldade em um jogo tende a ser um elemento extremamente visível e importante para o seu aproveitamento. Jogos mais desafiadores são conhecidos por serem mais frustrantes, e os menos desafiadores, mais relaxantes. Porém, esse pensamento pode levar à falsa idéia de impossibilitação do aproveitamento por parte da dificuldade elevada. No artigo de Jesper~\citet{FearOfFailing} é discutido o verdadeiro papel da dificuldade em jogos, de forma enxuta e didática. Quando apresentamos o conceito de vitória em um jogo, não faria sentido termos este sem o conceito de derrota. A derrota entra na experiência do jogo de forma a induzir os jogadores a reajustarem sua perspectiva perante ao jogo, como visto no estudo de Juul, levando o jogador a sempre pensar em novas estratégias e, consequentemente, adicionando conteúdo ao jogo.

O nivel de dificuldade de um jogo pode ser visto como o responsável pelo desenvolvimento nas habilidades do jogador em experimentos como o de~\citet{ExperimentalValidation}, porém, neste mesmo experimento, vemos ela também influenciando fortemente o aproveitamento geral do jogo, e podendo prejudicá-lo se não definida apropriadamente. Segundo~\citet{VideoGameBusiness}, jogos são uma das formas de entretenimento mais consumidas atualmente, totalizando uma receita de 159.3 bilhões de dólares no mundo inteiro em 2020. Portanto é natural associarmos jogos a entretenimento e, consequentemente, imaginarmos que um dado jogo é divertido, porém adequar a dificuldade de um jogo à experiência desejada prova-se um trabalho difícil.

escrever mais aqui
\textquotedbl{aspa}\textquotedbl{}

\section{Objetivos}
