%!TeX root=../tese.tex
%("dica" para o editor de texto: este arquivo é parte de um documento maior)
% para saber mais: https://tex.stackexchange.com/q/78101/183146

%% ------------------------------------------------------------------------- %%
\chapter{Dificuldade e Gêneros de Jogos}
\label{cap:dificuldade e generos de jogos}
\section{Arcades e Consoles}

\textit{Arcades}, mais comumente conhecidos como \textit{Fliperamas}, foram introduzidas na década de 1930, com jogos mecânicos ainda hoje famosos, como \textit{Pinball}. As máquinas eram projetadas com uma entrada de moeda, que quando inserida permitia que fossem jogados determinado número de vezes (\textquotedbl{rounds}\textquotedbl{} ou \textquotedbl{vidas}\textquotedbl{}). Não se afastando desse sistema, os jogos eletrônicos em gabinetes foram introduzidos apenas na década de 60 às \textit{Arcades}. Como relatado por~\citet{ArcadeGaming}, jogos como \textit{Pac-Man}, \textit{Donkey Kong}, \textit{Space Invaders}, \textit{Asteroids} e diversos outros títulos dominavam as \textit{Arcades} no mundo inteiro.

A partir desse ponto, foram apenas introduzidas novas tecnologias aos jogos de \textit{Arcade}. \textit{Duck Hunt} foi considerado revolucionário por usar um controle no formato  de um revólver, usado para mirar diretamente na tela, simulando um tiro-ao-alvo. Alguns anos depois, jogos de corrida como \textit{Grand Prix} levaram desenvolvedores à ideia de um controle no formato de um volante e um pedal, simulando, similarmente ao \textit{Duck Hunt}, um cenário de pilotagem mais próximo da realidade.

Por mais atraentes que os jogos de \textit{Arcade} estavam se tornando, sua dificuldade tendia a surpreender mais ainda o jogador. Até este momento, o papel principal da dificuldade era de se aproveitar ao máximo do sistema de moedas do jogo. Quanto mais difícil o jogo, mais moedas seriam gastas para completá-lo, e poucos desenvolvedores deixavam de ver essa vantagem.

Com a entrada dos \textit{consoles}\footnote{
    Aparelho dedicado a rodar um determinado grupo de jogos, com hardwares de interface (ex.: Playstation 2, Nintendo GameBoy, Xbox360).
}, principalmente do \textit{Nintendo Entertainment System}, também conhecido como \textit{NES} e \textit{Nintendinho}, jogos começaram a ser reestruturados, deixando de ter apenas uma fase e seu objetivo de arrecadar o maior número de pontos possível, e adquirindo uma estrutura mais linear, com diversas fases e, consequentemente, tendo seu objetivo principal deslocado para terminar todas as fases, ou derrotar o último chefe. Com isso, o papel da dificuldade sofreu uma leve mudança, deixando de buscar as moedas do jogador, e passando a ser a sustentação principal do tempo de jogo. Quanto mais difícil, mais tempo o jogador levaria para terminar o jogo, portanto maior aproveitamento seria tirado. Ideologias como essa moldaram a grande maioria dos jogos de \textit{NES} como jogos extremamente difíceis de se completar.

\section{Cenário Atual}

Após uma onda de jogos notoriamente difíceis, desenvolvedores começaram a explorar a ideia de dificuldades selecionáveis em jogos, em busca de poderem atender públicos mais abrangentes com seus projetos. Estava então emergindo a mecânica de, ao começar um \textit{novo jogo}, era apresentada a opção de dificuldade, normalmente variando entre \textit{fácil}, \textit{médio} e \textit{difícil}, mas não se limitando a essas opções.

Variáveis do jogo como quantidade de pontos de vida do jogador e dos inimigos oscilavam de acordo com a dificuldade escolhida. Os primeiros grandes jogos de sucesso a explorar mais a fundo a mecânica de diversos níveis de dificuldade, como \textit{Halo}, \textit{Call of Duty}, saga \textit{The Elder Scrolls}, saga \textit{Final Fantasy}, entre outros, moldaram esse padrão para uma grande parcela dos jogos futuros, porém não a perfeccionando instantaneamente. Picos na dificuldade eram comumente encontrados em níveis mais altos no jogo. As últimas dificuldades, normalmente rotuladas como \textit{Impossível} ou termos do gênero, costumavam inflar exageradamente variáveis do jogo como pontos de vida dos inimigos, muitas vezes tornando o jogo praticamente impossível e distanciando o jogador do cenário de entretenimento que os desenvolvedores se propunham a oferecer.

Em paralelo, é possível observar certas estruturas de jogos que permitem ao jogador explorar suas mecânicas de modo a afetar a dificuldade. Por exemplo, jogos no estilo \textit{Online Multiplayer}\footnote{
    Em contrapartida ao \textit{Singleplayer}, é um estilo de jogo onde diversas pessoas jogam uma partida juntas, em tempo real, através de seus respectivos aparelhos ou \textit{consoles} e uma conexão à \textit{internet}.
}, jogadores frequentemente participam de partidas e deliberadamente evitam usar certos itens do jogo, de modo a propor um desafio maior a si mesmos, se exibir ou até mesmo estabelecer recordes. Neste último cenário, também vemos a grande presença das \textit{Speedruns} em jogos \textit{Singleplayer}. Estas consistem em o jogador tentar completar determinado jogo no período de tempo mais curto possível. Para isso, o jogo é estudado profundamente e suas estratégias são otimizadas, procurando pular fases, explorar pontos fracos de inimigos ou até mesmo \textit{bugs} nos jogos. Essa cultura é popularmente conhecida por propor tarefas de dificuldade significativamente mais elevadas que o jogo \textquotedbl{base}\textquotedbl{}, consequentemente demandando muito da habilidade do jogador, ao mesmo tempo que adiciona conteúdo ao jogo.

Com o decorrer das décadas de 2000 e 2010, desenvolvedores começaram a aprofundar a complexidade do balanceamento da dificuldade de seus jogos~\citep{Zork:81}. A introdução de \textit{checkpoints}\footnote{
    Quando encontrado, garante que o jogador possa voltar a este ponto do jogo após ser derrotado. Funciona normalmente como uma \textquotedbl{volta no tempo}\textquotedbl{} para não punir tão severamente o jogador.
} tornou jogos extensos muito mais acessíveis~\citep{GameTime}, por impedir que a derrota levasse o jogador de volta para o primeiro nível do jogo, e reduzindo o quanto do jogo teria que ser rejogado. Jogos mais datados são conhecidos por praticamente não possuírem checkpoints, o que contribuia fortemente para sua dificuldade elevada. É importante, porém, manter em mente que um uso excessivo de \textit{checkpoints} pode trivializar as conquistas do jogador, uma vez que este não será apropriadamente punido por seus erros.

Neste mesmo período, com o avanço das Inteligências Artificiais (IAs), jogos \textit{Singleplayer}\footnote{
    Estilo de jogo projetado para ser jogado por uma única pessoa simultaneamente.
} introduziram novas dinâmicas, não apenas de dificuldade, mas de jogabilidade no geral. Um dos gêneros que mais se benficiou desse avançou foi o de jogos de luta. Originados da geração dos \textit{arcades}, jogos de luta eram projetados no estilo \textit{versus} para dois ou mais jogadores invariavelmente. Porém, com a entrada das IAs, foi possibilitado o modo de jogo para um único jogador, onde este enfrentaria um ou mais personagens controlados pelo próprio jogo, e, em muitos casos, com níveis de dificuldade selecionáveis. Com isso, uma grande gama de habilidades poderia ser atendida pelo jogo, ao mesmo tempo que era reduzida a necessidade da presença de outro jogador para que o jogo pudesse ser aproveitado ao máximo.

O recém-lançado \textit{Risk of Rain 2}, desenvolvido pela \textit{Hopoo Games}, tornou-se uma referência interessante para os estudos de dificuldade de jogos eletrônicos. Trata-se de um \textit{Rogue-Like}\footnote{
    Gênero de jogo onde os mapas são, normalmente, gerados aleatória ou proceduralmente. A derrota do jogador o força a recomeçar o jogo desde o primeiro mapa, porém permitindo que sejam desbloqueados elementos novos para o jogo neste processo.
} onde a dificuldade é aumentada com o decorrer do tempo, independente das ações do jogador, que por sua vez fica encarregado de \textquotedbl{acompanhar}\textquotedbl{} a dificuldade do jogo conforme esta avança. A progressão da dificuldade é baseada em valores como tempo de jogo, quantidade de jogadores, fase atual do jogador, entre outros valores, cuja fórmula será aprofundada em um capítulo mais adiante.

\section{Gênero Escolhido}\label{genero}

Com os avanços não apenas na tecnologia disponível para desenvolvimento de jogos, mas também na criatividade dos desenvolvedores, jogos eletrônicos começaram a tomar uma complexidade que tornou inevitável uma segmentação destes em gêneros. Atualmente, o gênero de um jogo não necessariamente definirá seu nível de dificuldade. Em grande parte dos gêneros de jogos podemos encontrar tanto jogos difíceis como fáceis. Porém, existe um gênero que se destaca entre os outros por englobar alguns dos jogos mais difíceis conhecidos, portanto o protótipo de jogo desenvolvido neste estudo seguiu este modelo de jogo.

O \textit{Bullet Hell}, também conhecido como \textit{Shoot 'em Up}, ou apenas \textit{Shmup}, é um subgênero de jogos eletrônicos de tiro em 2D\footnote{
    Em um jogo 2D, o jogador controla a posição do personagem em apenas duas dimensões. No caso dos \textit{Shmup}s, o personagem se move apenas para cima, para baixo, esquerda e direita em um plano.
}, onde o jogador controla um único personagem, normalmente uma espaçonave ou avião, este disparando projéteis em diversos inimigos, em uma única direção, enquanto foca paralelamente em desviar de uma quantidade demasiada de projéteis vindos de tais inimigos~\citep{STG}. Em grande parte dos títulos neste gênero, o jogador, uma vez que acertado por um tiro, perde qualquer poder que tenha acumulado ao longo da fase, e uma vida. Quando o seu número de vidas chega em 0, ele perde o jogo totalmente.

Uma das definições populares para um shmup, como discutido por~\citet{DosNDontsShmups}, engloba o conceito de não permitir que o jogador mate muitos inimigos sem precisar se preocupar com desviar de seus tiros. Grande parte da experiência de um \textit{Shmup} se encontra na necessidade de concentração para desviar de todos os projéteis que apresentam perigo ao jogador. Isso é, trata-se de jogos onde o usuário dificilmente se sentirá superior aos inimigos, constantemente adaptando suas estratégias para superar os grandes desafios proporcionados. Nexic também menciona que, se este conforto for dado ao jogador, este estará mais propício a se frustrar quando for atingido por uma \textquotedbl{bala perdida}\textquotedbl{}.

O gênero teve sua origem com o famoso \textit{Space Invaders} em 1978 e seus sucessores \textit{Galaxian} e \textit{Defender}, em 1980, mas sua emersão foi com a série \textit{Radiant Silvergun} em 1998~\citep{Geemu}. Possibilitado pelo avanço na capacidade computacional tanto de computadores pessoais como de \textit{consoles}, novas mecânicas e visuais foram introduzidos ao gênero rapidamente. Apesar de jogos no subgênero de \textit{Shmups} terem sido e ainda serem classificados como jogos de nicho, a trajetória de seu avanço os tornou significativamente únicos e realçados dentro do gênero de jogos de tiro.

Os jogos que serviram como a maior inspiração para este estudo foram os da série \textit{Touhou Project}, esta contendo mais de 20 jogos, desenvolvida únicamente por Jun'ya Ota, mais conhecido como \textit{ZUN}~\citep{Touhou}. Apesar de alguns jogos da série seguirem o gênero de jogos de luta, a grande maioria não apenas se enquadra no gênero de \textit{Shmup} como também serve de referência até os dias de hoje. A enorme variedade de inimigos, personagens jogáveis, fases completas e percursos de treinamento encontrada na série foi o fator determinante para a escolha deste gênero para o estudo, uma vez que a criatividade define o limite de qualquer mecânica que venha a ser implementada. Poderes especiais dos personagens, padrões de projéteis de inimigos, padrões de movimentação de inimigos, interações dinâmicas de elementos~\citep{Ikaruga}, entre outras, são características que influenciam fortemente a dificuldade do jogo, de acordo com o modo como são implementadas. Este estudo procura atuar principalmente sobre esses aspectos do jogo para atingir seu objetivo de adaptação dinâmica de dificuldade.