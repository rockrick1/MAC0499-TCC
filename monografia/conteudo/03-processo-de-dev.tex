%!TeX root=../tese.tex
%("dica" para o editor de texto: este arquivo é parte de um documento maior)
% para saber mais: https://tex.stackexchange.com/q/78101/183146

%% ------------------------------------------------------------------------- %%
\chapter{Processo de Desenvolvimento}
\label{cap:processo de desenvolvimento}
\section{Ideias Iniciais}

Inicialmente, foram avaliadas possibilidades de gêneros diversas para o jogo. Artigos como o de~\citet{DynamicDiffAdjustment} mostra as vantagens de uma abordagem baseada em colocar o jogador em uma situação onde enfrenta diversos inimigos, em oposição a um jogo de luta, onde ele poderia enfrentar apenas um, ou jogos de quebra-cabeça, onde normalmente não existem inimigos. Para isso, foi desenvolvido um pequeno protótipo no estilo \textit{top-down}\footnote{
    Jogos definidos pelo ângulo de visão do jogo, vendo o personagem por cima, em oposição à visão pelos lados, usada em jogos como \textit{Super Mario Bros}.
}, onde o personagem principal podia atirar em qualquer direção usando uma gama de armas selecináveis, em uma \textit{arena}. Isso é, ele se encontrava em uma fase fechada, com apenas algumas portas localizadas nos cantos, de onde os inimigos surgiam em ondas. Os inimigos então se direcionavam ao personagem, causando dano caso entrassem em contato com ele. Inimigos de tipos diferentes também foram desenvolvidos, como um personagem que se aproximava do jogador, e quando dentro de uma certa distância dele, atirava projéteis em sua direção.

Conforme estes tipos diferentes ameaças eram introduzidas, a ideia para o cálculo da métrica de dificuldade apresentou-se cada vez mais como um desafio. Não apenas as ameaças, mas o nível de liberdade que o jogador possuia para se movimentar pela fase, atirar em qualquer direção e se aproveitar da arquitetura das arenas inseriu muitas variáveis a serem consideradas na parametrização da dificuldade, tanto para leitura, avaliação dos movimentos do jogador e seu desempenho, quanto para aplicação de balanceamento, alteração de variáveis dos inimigos e até mesmo do personagem principal.

Com isso, a exploração de outro gênero tornou-se um caminho mais apropriado para os objetivos deste estudo. A simplicidade dos \textit{Shmups}, quando comparados aos jogos de \textit{Arena} em qualquer estilo, provou facilitar consideravelmente esta interpretação de ações do jogador para formação da métrica de dificuldade. Não se limitando a isso, porém, foi possível também generalizar o modelo dos inimigos de forma que, lendo essa métrica gerada, entre outras variáveis, o nível do inimigo se adapta em tempo real, tanto quando a dificuldade é aumentada, quanto diminuída.

\section{Decisões de \textit{Design}}

Em um cenário de desenvolvimento de jogos, é necessário definir firmemente os conceitos fundamentais do jogo, não apenas de um ponto de vista de \textit{design}, pensando nos objetivos do jogo, quais mecânicas serão implementadas, mas também de um ponto de vista técnico, dessa vez pensando em que ferramentas utilizar e tipos de arquitetura de jogos para usar como base. Esta seção será dedicada a essas especificações funcionais e técnicas do jogo.

\subsection{Estrutura Geral}

Para entendimento da estrutura geral do jogo, utiliza-se a metodologia de \textit{cenas}. Uma cena pode ser interpretada de formas diferentes dentro do jogo. Uma boa maneira de se visualizá-las é pensando em um \textquotedbl{estado de jogo}\textquotedbl{}. Por exemplo, o \textit{menu} principal do jogo terá uma cena dedicada apenas a ele, assim como o local onde o jogo se passa, neste caso, no espaço, também está separado em uma cena específica. Os tipos diferentes de cenas e suas funcionalidades serão aprofundadas em um capitulo mais adiante.

O menu principal do jogo possui as opções de \textit{Novo Jogo}, iniciando assim um jogo no modo padrão, logo após uma tela de seleção de personagens; \textit{Tutorial}, colocando o jogador em uma fase guiada por instruções de como jogar, quais botões utilizar e explicando todas as mecânicas implementadas que o jogador precisa saber; \textit{Demo}, levando-o a uma demonstração de que tipos de padrões de tiros os inimigos podem gerar, o qual será aprofundado em um capitulo futuro também; e por fim, \textit{Sair}, que fecha o jogo.

falar mais aqui talvez?

\subsection{Progressão e Ondas}

Ao iniciar um \textit{Novo Jogo}, o jogador depara-se com a tela de seleção de personagens (naves) mecionada anteriormente. Ao escolher entre uma das naves, uma tela contendo apenas a nave escolhida, o fundo temático do espaço sideral e indicadores da quantidade de vidas e de bombas do jogador é apresentada na sequência. O jogador pode se mover livremente dentro das delimitações laterais da tela e atirar sem se preocupar com sua munição. Após alguns segundos, os primeiros inimigos aparecerão pelo canto superior da tela. Estes fazem parte da primeira onda de inimigos. Os inimigos então começarão a disparar projéteis, diretamente em direção à nave principal ou não, de maneiras pré programadas. Cada inimigo permanece um tempo determinado na tela antes de ser arrastado para sua lateral mais próxima e \textit{despawnando}\footnote{
    Em oposição ao \textit{spawn}, que significa o \textit{inserir} de uma entidade no estado atual do jogo, o \textit{despawn} é quando uma entidade é liberada e deletada da cena atual do jogo.
}, caso não seja derrotado pelo jogador antes. Quando todos os inimigos de uma \textit{onda} são derrotados ou \textit{despawnados}, é iniciada a onda seguinte de inimigos, podendo conter inimigos mais fortes, mais fracos, em maior ou menor quantidade, parâmetro a ser definido no banco de dados do jogo.

Ao chegar na última onda de inimigos, o jogador é apresentado ao chefe principal da fase. O chefe possui uma quantidade considerável a mais de pontos de vida, tornando mais difícil de matar, assim como ataques bastante distintos dos de inimigos comuns, e também em maior variedade. O chefe também pode se movimentar uniformemente pelo topo da fase, de forma a forçar o deslocamento do jogador caso este esteja focando seu posicionamento em um único ponto. Podem também haver \textit{sub-chefes} ou \textit{mini-chefes} antes da onda final da fase, estes proporcionando uma batalha similar à com o chefe final, porém em uma dificuldade levemente menos elevada. Ao derrotar o chefe final, a fase atual acaba e o jogador vence o jogo após ser parabenizado.

\subsection{Poderes Especiais}

Ao entrar na fase, o jogador terá em seu arsenal dois principais meios de combate aos inimigos. O mais usado será o seu tiro primário. Como mencionado anteriormente, este não custará munição, portanto poderá ser usado a vontade. Ao derrotar um inimigo, este depositará na fase uma quantidade determinada de \textit{drops}\footnote{
    Em jogos, quando um inimigo deixa para trás algum item ao morrer, este item é denominado \textit{drop}.
} de cristais, que quando coletados pelo jogador, somarão um número aleatório, em um intervalo específico, de pontos para o jogador. Ao juntar uma quantidade de pontos, terá o \textit{nível} de seu tiro primário aumentado, tornando-o mais forte, com alcance maior e, assim, mais versátil e ameaçador.

O estilo do tiro primário varia de acordo com a escolha de nave, isso é, tanto o primeiro nível do tiro quanto os demais serão diferente para cada nave implementada. Isso disponibiliza diversas opções de jogabilidade, ao mesmo tempo que não causa um impacto severo na dificuldade do jogo nem na sua complexidade, permitindo também que as mecânicas de balanceamento se mantenham estabilizadas.

O outro método de ataque disponível é a bomba. Esta, por sua vez, não será tão abundante quanto o tiro primário. Como notável pelo jogador assim que a fase é iniciada, há um contador de bombas disponíveis junto ao número de vidas. Quando uma bomba é usada, esse contador é decrementado uma unidade, e se não houverem bombas disponíveis, o jogador terá que continuar derrotando inimigos utilizando apenas seu tiro primário como ataque antes de poder usá-las novamente. Similarmente à progressão de níveis do tiro primário, ao coletar um determinado número de cristais de inimigos, o jogador ganhará uma bomba. Quando usada, a bomba causa uma explosão visível, causando dano massivo a todos os inimigos instantaneamente, e também limpando quaisquer projéteis inimigos que estiverem na tela. Com isso, a bomba se torna um recurso valioso que deve ser usado estrategicamente, possivelmente sendo guardada para o encontro com o chefe da fase.

Por último, foi implementada uma habilidade muito comumente encontrada em \textit{Shmups}, conhecida como \textit{Strafe}, ou \textit{Focus}, assim chamada em \textit{Touhou Project}. Enquanto o botão designado ao \textit{Strafe} estiver pressionado, a nave do jogador terá sua velocidade reduzida, e um pequeno indicador da \textit{hitbox}\footnote{
    Nome dado à zona de colisão de um objeto com outros em jogos. Isso vale tanto para a nave principal quanto para os inimigos. Quando um projétil entra na \textit{hitbox} do jogador ou de um inimigo, este levará o dano no tiro.
} da nave se tornará aparente. Ao soltar o botão, a velocidade da nave volta ao normal e o indicador da \textit{hitobox} desaparece. Isso se prova útil não apenas para desviar de uma leva densa de balas com mais precisão, mas também para tornar evidente a verdadeira distância de perigo das balas, uma vez que, em grande parte dos \textit{Shmups}, a \textit{hitobox} da nave principal é consideravelmente menor do que a sua \textit{sprite}\footnote{
    Imagem do objeto desenhada na tela. No caso do personagem principal, é o desenho de sua nave.
}, dificultando a tarefa de desviar das balas inimigas quando este indicador não está visível.

\subsection{Visuais e Sons}

Indicadores visuais e/ou sonoros se tornam extremamente úteis ao permitir que o jogo expresse eventos que, de outra forma, não seriam notados pelo jogador.

indicadores de qualquer coisa, mencionar a dificuldade depois só
inimigos diferentes, balas maiores, sons e musica?? ofuck

