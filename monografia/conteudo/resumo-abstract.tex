%!TeX root=../tese.tex
%("dica" para o editor de texto: este arquivo é parte de um documento maior)
% para saber mais: https://tex.stackexchange.com/q/78101/183146

% O resumo é obrigatório, em português e inglês. Este comando também gera
% automaticamente a referência para o próprio documento, conforme as normas
% sugeridas da USP
\begin{resumo}{port}
O objetivo deste trabalho foi estudar e implementar o balanceamento dinâmico
de dificuldade em jogos eletrônicos. Foi desenvolvido um protótipo de um jogo
no estilo bullet-hell. Dada a infame e elevada dificuldade encontrada
neste estilo de jogo, o foco principal do estudo foi de torná-lo
acessível a todos os públicos, independente da habilidade do jogador. Foram
adotadas funções matemáticas que levam em consideração diversas variáveis
do jogo e definem uma métrica de dificuldade a ser usada no balanceamento de
inimigos e recursos em tempo real. Foi enfim criado um ciclo de jogabilidade
capaz de se adaptar às ações do jogador e dirigir o nível de dificuldade
do jogo, procurando otimizar o entretenimento proposto.
\end{resumo}


% O resumo é obrigatório, em português e inglês. Este comando também gera
% automaticamente a referência para o próprio documento, conforme as normas
% sugeridas da USP
\begin{resumo}{eng}
The main goal of this thesis was to study and implement the dynamic balancing
of difficulty in video games. With that purpose in mind, a bullet-hell style
game prototype was developed. Given the infamous high difficulty in this
game genre, the main focus of the study was to make it as accessible to all
publics as possible, regardless of the players ability. Mathematical funcions
were used to consider many game variables and define a difficulty metric to
be later used in the balancing of enemy difficulty and resource abundancy,
all in gameplay time. Thus, a full cycle of playability was developed,
capable of adapting to the player's actions and steer the level of difficulty
of the game, aiming to optimize the enjoyment.
\end{resumo}



% O objetivo deste trabalho foi estudar e implementar o balanceamento dinâmico
% de dificuldade em jogos eletrônicos. Foi desenvolvido um protótipo de um jogo
% no estilo bullet-hell. Dada a infame e elevada dificuldade encontrada
% neste estilo de jogo, o foco principal do estudo foi de torná-lo
% acessível a todos os públicos, independente da habilidade do jogador, por
% meio da parametrização da habilidade medida à dificuldade do jogo. Foram
% adotadas funções matemáticas que levam em consideração diversas variáveis
% do jogo e definem uma métrica de dificuldade, a ser usada no balanceamento de
% inimigos e recursos. Foi enfim criado um ciclo de jogabilidade capaz
% de se adaptar às ações do jogador e dirigir o nível de dificuldade do jogo,
% procurando otimizar o entretenimento proposto.